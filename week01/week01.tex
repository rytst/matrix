\documentclass{article}
\usepackage{amsmath}
\usepackage{amsthm}
\usepackage{amsfonts}
\usepackage{mathtools}
\usepackage{mathrsfs}
\usepackage{ascmac}
\usepackage{bm}
\theoremstyle{plain}
\newtheorem{dfn}{Definition}[subsection]
\title{Matrix Algebra Marathon}
\author{J2200071 Ryuto Saito}
\date{\today}

\begin{document}
\maketitle

\section{Inner-Product of Vectors}

\subsection{Definition}

\begin{dfn}
  The inner-product of vectors is defined as
  \begin{equation}
    \label{def_inner}
    \langle \bm{x} , \bm{y} \rangle \coloneq \sum_{i=1}^n x_i y_i
  \end{equation}
  where
  \begin{math}
    \bm{x} , \bm{y} \in \mathbb{R}^n .
  \end{math}
\end{dfn}


\subsection{Exercise}

\begin{itembox}[l]{Exercise 2.1.3.}
  For
  \begin{math}
    \forall a \in \mathbb{R}
  \end{math}
  and
  \begin{math}
    \forall \bm{x} , \forall \bm{y} \in \mathbb{R}^n ,
  \end{math}
  show that
  \begin{equation}
    \label{ex213}
    \langle a \bm{x} , \bm{y} \rangle = a \langle \bm{x} , \bm{y} \rangle = \langle \bm{x} , a \bm{y} \rangle .
  \end{equation}
\end{itembox}

\begin{proof}
  For all
  \begin{math}
    a \in \mathbb{R}
  \end{math}
  and
  \begin{math}
    \bm{x} \in \mathbb{R}^n , \bm{y} \in \mathbb{R}^n ,
  \end{math}
  \begin{equation*}
    \langle a \bm{x} , \bm{y} \rangle = \sum_{i=1}^n (a x_i) y_i = a \sum_{i=1}^n x_i y_i = a \langle \bm{x} , \bm{y} \rangle
  \end{equation*}
  and
  \begin{equation*}
    \langle \bm{x} , a \bm{y} \rangle = \sum_{i=1}^n x_i (a y_i) = a \sum_{i=1}^n x_i y_i = a \langle \bm{x} , \bm{y} \rangle .
  \end{equation*}
  Thus,
  \begin{equation*}
    \langle a \bm{x} , \bm{y} \rangle = a \langle \bm{x} , \bm{y} \rangle = \langle \bm{x} , a \bm{y} \rangle .
  \end{equation*}
\end{proof}


\begin{itembox}[l]{Exercise 2.1.6.}
  For
  \begin{math}
    \forall \bm{x} , \forall \bm{y} , \forall \bm{z} \in \mathbb{R}^n ,
  \end{math}
  show that
  \begin{equation}
    \label{ex216}
    \langle \bm{x} , \bm{y} + \bm{z} \rangle = \langle \bm{x} , \bm{y} \rangle + \langle \bm{x} , \bm{z} \rangle .
  \end{equation}
\end{itembox}

\begin{proof}
  For all
  \begin{math}
    \bm{x} , \bm{y} , \bm{z} \in \mathbb{R}^n ,
  \end{math}
  \begin{equation*}
    \begin{split}
      \langle \bm{x} , \bm{y} + \bm{z} \rangle &= \sum_{i=1}^n x_i (y_i + z_i) = \sum_{i=1}^n (x_i y_i + x_i z_i) = \sum_{i=1}^n x_i y_i + \sum_{i=1}^n x_i z_i \\
      &= \langle \bm{x} , \bm{y} \rangle + \langle \bm{x} , \bm{z} \rangle .
    \end{split}
  \end{equation*}
\end{proof}



\section{$\ell_2$-Norm}

\subsection{Definition}

\begin{dfn}
  The $\ell_2$-norm of vectors is defined as
  \begin{equation}
    \label{def_norm}
    \lVert \bm{x} \rVert \coloneq \sqrt{\langle \bm{x} , \bm{x} \rangle}
  \end{equation}
  where
  \begin{math}
    \bm{x} \in \mathbb{R}^n.
  \end{math}
\end{dfn}

\subsection{Exercise}

\begin{itembox}[l]{Exercise 2.2.4.}
  For
  \begin{math}
    \forall a \in \mathbb{R} , \forall \bm{x} \in \mathbb{R}^n ,
  \end{math}
  derive the equality:
  \begin{equation}
    \label{ex224}
    \lVert a \bm{x} \rVert = \lvert a \rvert \lVert \bm{x} \rVert
  \end{equation}
  where $\lvert a \rvert$ denotes the absolute value of a.
\end{itembox}

\begin{proof}
  For all
  \begin{math}
    a \in \mathbb{R}
  \end{math}
  and
  \begin{math}
    \bm{x} \in \mathbb{R}^n ,
  \end{math}
  \begin{equation*}
    \begin{split}
      \lVert a \bm{x} \rVert &= \sqrt{\langle a \bm{x} , a \bm{x} \rangle} = \sqrt{\sum_{i=1}^n (a x_i)(a x_i)} = \sqrt{a^2 \sum_{i=1}^n x_i^2} = \sqrt{a^2} \sqrt{\sum_{i=1}^n x_i^2} \\
      &= \lvert a \rvert \sqrt{\langle \bm{x} , \bm{x} \rangle} = \lvert a \rvert \Vert \bm{x} \rVert .
    \end{split}
  \end{equation*}
\end{proof}

\begin{itembox}[l]{Exercise 2.2.8.}
  Let 
  \begin{math}
    \bm{x} , \bm{y} \in \mathbb{R}^n .
  \end{math}
  Derive the equality:
  \begin{equation}
    \label{ex228}
    \lVert \bm{x} + \bm{y} \rVert^2 = \lVert \bm{x} \rVert^2 + \lVert \bm{y} \rVert^2 + 2 \langle \bm{x} , \bm{y} \rangle .
  \end{equation}
\end{itembox}

\begin{proof}
  Let
  \begin{math}
    \bm{x} , \bm{y} \in \mathbb{R}^n .
  \end{math}
  \begin{equation*}
    \begin{split}
      \lVert \bm{x} + \bm{y} \rVert^2 &= \left( \sqrt{\langle \bm{x} + \bm{y} , \bm{x} + \bm{y} \rangle} \right)^2 \\
      &= \langle \bm{x} + \bm{y} , \bm{x} + \bm{y} \rangle \\
      &= \sum_{i=1}^n (x_i + y_i)^2 \\
      &= \sum_{i=1}^n (x_i^2 + 2 x_i y_i + y_i^2) \\
      &= \sum_{i=1}^n x_i^2 + \sum_{i=1}^n 2 x_i y_i + \sum_{i=1}^n y_i^2 \\
      &= \sum_{i=1}^n x_i^2 + 2 \sum_{i=1}^n x_i y_i + \sum_{i=1}^n y_i^2 \\
      &= \langle \bm{x} , \bm{x} \rangle + \langle \bm{y} , \bm{y} \rangle + 2 \langle \bm{x} , \bm{y} \rangle \\
      &= \left( \sqrt{\langle \bm{x} , \bm{x} \rangle} \right)^2 + \left( \sqrt{\langle \bm{y} , \bm{y} \rangle} \right)^2 + 2 \langle \bm{x} , \bm{y} \rangle \\
      &= \lVert \bm{x} \rVert^2 + \lVert \bm{y} \rVert^2 + 2 \langle \bm{x} , \bm{y}\rangle
    \end{split}
  \end{equation*}
\end{proof}


\begin{itembox}[l]{Exercise 2.2.10.}
  Let
  \begin{math}
    \bm{x} , \bm{y} \in \mathbb{R}^n .
  \end{math}
  Show that
  \begin{equation}
    \label{ex2210}
    \langle \bm{x} , \bm{y} \rangle \leq \lVert \bm{x} \rVert \lVert \bm{y} \rVert .
  \end{equation}
\end{itembox}

\begin{proof}
  Let
  \begin{math}
    \bm{x} , \bm{y} \in \mathbb{R}^n , t \in \mathbb{R} .
  \end{math}
  Consider the function
  \begin{equation*}
    \begin{split}
      f(t) &= \lVert \bm{x} + t \bm{y} \rVert^2 \\
      &= \langle \bm{x} + t \bm{y} , \bm{x} + t \bm{y} \rangle \\
      &= \sum_{i=1}^n (x_i + t y_i)^2 \\
      &= \sum_{i=1}^n \left(x_i^2 + 2 t x_i y_i + (t y_i)^2\right) \\
      &= \sum_{i=1}^n x_i^2 + 2 t \sum_{i=1}^n x_i y_i + t^2 \sum_{i=1}^n y_i^2 \\
      &= \langle \bm{y} , \bm{y} \rangle t^2 + 2 \langle \bm{x} , \bm{y} \rangle t + \langle \bm{x} , \bm{x} \rangle \geq 0 .
    \end{split}
  \end{equation*}
  Therefore, the quadratic equation
  \begin{equation*}
    \langle \bm{y} , \bm{y} \rangle t^2 + 2 \langle \bm{x} , \bm{y} \rangle t + \langle \bm{x} , \bm{x} \rangle = 0
  \end{equation*}
  has at most one solution. This implies that its discriminant must be less or equal to zero, that is
  \begin{equation*}
    (2 \langle \bm{x} , \bm{y} \rangle)^2 - 4 \langle \bm{y} , \bm{y} \rangle \langle \bm{x} , \bm{x} \rangle \leq 0 .
  \end{equation*}
  Hence
  \begin{equation*}
    \begin{split}
      4 \langle \bm{x} , \bm{y} \rangle^2 &\leq 4 \langle \bm{x} , \bm{x} \rangle \langle \bm{y} , \bm{y} \rangle \\
      \langle \bm{x} , \bm{y} \rangle^2 &\leq \langle \bm{x} , \bm{x} \rangle \langle \bm{y} , \bm{y} \rangle \\
      \langle \bm{x} , \bm{y} \rangle^2 &\leq \left(\sqrt{\langle \bm{x} , \bm{x} \rangle}\right)^2 \left(\sqrt{\langle \bm{y} , \bm{y} \rangle}\right)^2 \\
      \langle \bm{x} , \bm{y} \rangle^2 &\leq \lVert \bm{x} \rVert^2 \lVert \bm{y} \rVert^2 ,
    \end{split}
  \end{equation*}
  so
  \begin{equation*}
    - \lVert \bm{x} \rVert \lVert \bm{y} \rVert \leq \langle \bm{x} , \bm{y} \rangle \leq \lVert \bm{x} \rVert \lVert \bm{y} \rVert .
  \end{equation*}
  This also implies
  \begin{equation*}
    \langle \bm{x} , \bm{y} \rangle \leq \lVert \bm{x} \rVert \lVert \bm{y} \rVert .
  \end{equation*}
\end{proof}


\begin{itembox}[l]{Exercise 2.2.14.}
  Let
  \begin{math}
    \bm{x} , \bm{y} \in \mathbb{R}^n
  \end{math}
  be unit vectors (i.e. $\lVert \bm{x} \rVert = \lVert \bm{y} \rVert = 1$) . Show that
  \begin{equation}
    \label{ex2214}
    \langle \bm{x} , \bm{y} \rangle \leq 1 .
  \end{equation}
\end{itembox}

\begin{proof}
  Let
  \begin{math}
    \bm{x} , \bm{y} \in \mathbb{R}^n
  \end{math}
  be unit vectors. It follows from (\ref{ex2210}) that
  \begin{equation*}
    \langle \bm{x} , \bm{y} \rangle \leq \lVert \bm{x} \rVert \lVert \bm{y} \rVert = 1 \cdot 1 = 1 .
  \end{equation*}
\end{proof}

\section{$\ell_1$-Norm of Vectors}

\subsection{Definition}

\begin{dfn}
  The $\ell_1$-norm of vectors is defined as
  \begin{equation}
    \label{def_l_1}
    \lVert \bm{x} \rVert_1 \coloneq \sum_{i=1}^n |x_i|
  \end{equation}
  where
  \begin{math}
    \bm{x} \in \mathbb{R}^n.
  \end{math}
\end{dfn}

\subsection{Exercise}

\begin{itembox}[l]{Exercise 2.3.3.}
  For
  \begin{math}
    \forall \bm{x} \in \mathbb{R}^n ,
  \end{math}
  show that
  \begin{equation}
    \label{ex233}
    \lVert \bm{x} \rVert_1 \geq 0 .
  \end{equation}
\end{itembox}

\begin{proof}
  Let $\bm{x} \in \mathbb{R}^n .$
  Since the absolute value is always either a positive number or zero, for all $i \in \mathbb{N}_n ,$
  \begin{equation*}
    |x_i| \geq 0 .
  \end{equation*}
  Then
  \begin{equation*}
    \lVert \bm{x} \rVert_1 = \sum_{i=1}^n |x_i| \geq 0 .
  \end{equation*}
\end{proof}


\section{$\ell_\infty$-Norm of Vectors}

\subsection{Definition}

\begin{dfn}
  The $\ell_\infty$-norm of vectors is defined as
  \begin{equation}
    \label{def_l_inf}
    \lVert \bm{x} \rVert_\infty \coloneq \max_{i \in \mathbb{N}_n} |x_i|
  \end{equation}
  where
  \begin{math}
    \bm{x} \in \mathbb{R}^n .
  \end{math}
\end{dfn}

\subsection{Exercise}

\begin{itembox}[l]{Exercise 2.4.3.}
  For
  \begin{math}
    \forall \bm{x} \in \mathbb{R}^n ,
  \end{math}
  show that
  \begin{equation}
    \label{ex243}
    \lVert \bm{x} \rVert_\infty \geq 0 .
  \end{equation}
\end{itembox}

\begin{proof}
  Let
  \begin{math}
    \bm{x} \in \mathbb{R}^n
  \end{math}
  and
  \begin{equation*}
    M = \max_{i \in \mathbb{N}_n} |x_i|.
  \end{equation*}
  Then, for all
  \begin{math}
    j \in \mathbb{N}_n ,
  \end{math}
  \begin{equation*}
    M \geq |x_j|
  \end{equation*}
  and
  \begin{equation*}
    |x_j| \geq 0 .
  \end{equation*}
  Thus,
  \begin{equation*}
    \lVert \bm{x} \rVert_\infty = \max_{i \in \mathbb{N}_n} |x_i| = M \geq |x_j| \geq 0 .
  \end{equation*}
\end{proof}



\end{document}
