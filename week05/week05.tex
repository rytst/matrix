\documentclass{article}
\usepackage{amsmath}
\usepackage{amsthm}
\usepackage{amsfonts}
\usepackage{mathtools}
\usepackage{mathrsfs}
\usepackage{ascmac}
\usepackage{bm}
\theoremstyle{plain}
\newtheorem{dfn}{Definition}[subsection]
\title{Matrix Algebra Marathon}
\author{J2200071 Ryuto Saito}
\date{\today}

\begin{document}
\maketitle

\section{Derivatives of Function of Matrices}


\subsection{Definition}

\begin{dfn}
	We denote the derivative of a function $f:\mathbb{R}^{m \times n} \rightarrow \mathbb{R}$ by
	\begin{equation*}
		\nabla_{\bm{X}} f(\bm{X}) \coloneq
		\begin{bmatrix}
			\nabla_{X_{1,1}} (f(\bm{X})) & \cdots & \nabla_{X_{1,n}} (f(\bm{X})) \\
			\vdots & \ddots & \vdots \\
			\nabla_{X_{m,1}} (f(\bm{X})) & \cdots & \nabla_{X_{m,n}} (f(\bm{X}))
		\end{bmatrix} . \\
	\end{equation*}
\end{dfn}


\subsection{Exercise}


\begin{itembox}[l]{Exercise 2.14.2.}
	Let
	\begin{math}
		\bm{X} , \bm{A} \in \mathbb{R}^{m \times n} .
	\end{math}
	Show that
	\begin{equation}
		\nabla_{\bm{X}} (tr(\bm{A}^{\mathrm{T}} \bm{X})) = \bm{A} .
	\end{equation}
\end{itembox}


\begin{proof}
	Let
	\begin{math}
		\bm{X} , \bm{A} \in \mathbb{R}^{m \times n} .
	\end{math}
	Recall the definition of inner-product of matrices and
	\begin{equation*}
		\langle \bm{X} , \bm{Y} \rangle = tr(\bm{X}^\mathrm{T} \bm{Y}) .
	\end{equation*}
	\begin{equation*}
		\begin{split}
			\nabla_{\bm{X}} (tr(\bm{A}^{\mathrm{T}} \bm{X})) &=
			\begin{bmatrix}
				\nabla_{X_{1,1}} tr(\bm{A}^\mathrm{T} \bm{X}) & \cdots & \nabla_{X_{1,n}} tr(\bm{A}^\mathrm{T} \bm{X}) \\
				\vdots & \ddots & \vdots \\
				\nabla_{X_{m,1}} tr(\bm{A}^\mathrm{T} \bm{X}) & \cdots & \nabla_{X_{m,n}} tr(\bm{A}^\mathrm{T} \bm{X})
			\end{bmatrix} \\
			&=
			\begin{bmatrix}
				\nabla_{X_{1,1}} \left( \sum_{i=1}^m \sum_{j=1}^n A_{i,j} X_{i,j} \right) & \cdots & \nabla_{X_{1,n}} \left( \sum_{i=1}^m \sum_{j=1}^n A_{i,j} X_{i,j} \right) \\
				\vdots & \ddots & \vdots \\
				\nabla_{X_{m,1}} \left( \sum_{i=1}^m \sum_{j=1}^n A_{i,j} X_{i,j} \right) & \cdots & \nabla_{X_{m,n}} \left( \sum_{i=1}^m \sum_{j=1}^n A_{i,j} X_{i,j} \right)
			\end{bmatrix} \\
			&=
			\begin{bmatrix}
				A_{1,1} & \cdots & A_{1,n} \\
				\vdots & \ddots & \vdots \\
				A_{m,1} & \cdots & A_{m,n}
			\end{bmatrix} \\
			&= \bm{A}
		\end{split}
	\end{equation*}
\end{proof}


\begin{itembox}[l]{Exercise 2.14.4.}
	Let
	\begin{math}
		\bm{X} , \bm{A} , \bm{B} \in \mathbb{R}^{m \times n} .
	\end{math}
	Show that
	\begin{equation}
		\nabla_{\bm{X}} (tr(\bm{A}^\mathrm{T} (\bm{X} + \bm{B}))) = \bm{A} .
	\end{equation}
\end{itembox}


\begin{proof}
	Let
	\begin{math}
		\bm{X} , \bm{A} , \bm{B} \in \mathbb{R}^{m \times n} .
	\end{math}
	\begin{equation*}
		\begin{split}
			&\nabla_{\bm{X}} (tr(\bm{A}^{\mathrm{T}} (\bm{X} + \bm{B}))) \\
			&=
			\begin{bmatrix}
				\nabla_{X_{1,1}} (tr(\bm{A}^{\mathrm{T}} (\bm{X} + \bm{B}))) & \cdots & \nabla_{X_{1,n}} (tr(\bm{A}^{\mathrm{T}} (\bm{X} + \bm{B}))) \\
				\vdots & \ddots & \vdots \\
				\nabla_{X_{m,1}} (tr(\bm{A}^{\mathrm{T}} (\bm{X} + \bm{B}))) & \cdots & \nabla_{X_{m,n}} (tr(\bm{A}^{\mathrm{T}} (\bm{X} + \bm{B}))) \\
			\end{bmatrix} \\
			&=
			\begin{bmatrix}
				\nabla_{X_{1,1}} \left( \sum_{i=1}^m \sum_{j=1}^n A_{i,j} (X_{i,j} + B_{i,j}) \right) & \cdots & \nabla_{X_{1,n}} \left( \sum_{i=1}^m \sum_{j=1}^n A_{i,j} (X_{i,j} + B_{i,j}) \right) \\
				\vdots & \ddots & \vdots \\
				\nabla_{X_{m,1}} \left( \sum_{i=1}^m \sum_{j=1}^n A_{i,j} (X_{i,j} + B_{i,j}) \right) & \cdots & \nabla_{X_{m,n}} \left( \sum_{i=1}^m \sum_{j=1}^n A_{i,j} (X_{i,j} + B_{i,j}) \right) \\
			\end{bmatrix} \\
			&=
			\begin{bmatrix}
				A_{1,1} & \cdots & A_{1,n} \\
				\vdots & \ddots & \vdots \\
				A_{m,1} & \cdots & A_{m,n}
			\end{bmatrix} \\
			&= \bm{A}
		\end{split}
	\end{equation*}
\end{proof}


\section{Derivatives of Matrix-Valued Functions}


\subsection{Definition}

\begin{dfn}
	Consider a matrix-valued function $\bm{F}:\mathbb{R} \rightarrow \mathbb{R}^{m \times n}$
	expressed as
	\begin{equation*}
		\bm{F}(x) \coloneq
		\begin{bmatrix}
			F_{1,1}(x) & \cdots & F_{1,n}(x) \\
			\vdots & \ddots & \vdots \\
			F_{m,1}(x) & \cdots & F_{m,n}(x)
		\end{bmatrix} .
	\end{equation*}
	The derivative of the function is denoted by
	\begin{equation*}
		\nabla_{x} \bm{F}(x) \coloneq
		\begin{bmatrix}
			\nabla_x F_{1,1}(x) & \cdots & \nabla_x F_{1,n}(x) \\
			\vdots & \ddots & \vdots \\
			\nabla_x F_{m,1}(x) & \cdots & \nabla_x F_{m,n}(x)
		\end{bmatrix} .
	\end{equation*}
\end{dfn}


\subsection{Exercise}


\begin{itembox}[l]{Exercise 2.15.3.}
	Let
	\begin{math}
		\bm{A} \in \mathbb{R}^{m \times n}
	\end{math}
	and
	\begin{math}
		\bm{F}:\mathbb{R} \rightarrow \mathbb{R}^{m \times n} .
	\end{math}
	Show that
	\begin{equation}
		\nabla_x \langle \bm{A} , \bm{F}(x) \rangle = \langle \bm{A} , \nabla_x \bm{F}(x) \rangle .
	\end{equation}
\end{itembox}


\begin{proof}
	Let
	\begin{math}
		\bm{A} \in \mathbb{R}^{m \times n}
	\end{math}
	and
	\begin{math}
		\bm{F}:\mathbb{R} \rightarrow \mathbb{R}^{m \times n} .
	\end{math}
	\begin{equation*}
		\begin{split}
			\nabla_x \langle \bm{A} , \bm{F}(x) \rangle &= \nabla_x \left( \sum_{i=1}^m \sum_{j=1}^n A_{i,j} F_{i,j}(x) \right) \\
			&= \left( \sum_{i=1}^m \sum_{j=1}^n A_{i,j} \nabla_x  F_{i,j}(x) \right) \\
			&= \langle \bm{A} , \nabla_x \bm{F}(x) \rangle
		\end{split}
	\end{equation*}
\end{proof}


\begin{itembox}[l]{Exercise 2.15.5.}
	Let $\bm{F}:\mathbb{R} \rightarrow \mathbb{R}^{m \times k}$
	and $\bm{G}:\mathbb{R} \rightarrow \mathbb{R}^{k \times n}$. Show that
	\begin{equation}
		\label{ex2155}
		\nabla_x (\bm{F}(x) \bm{G}(x)) = (\nabla_x \bm{F}(x)) \bm{G}(x) + \bm{F}(x) \nabla_x \bm{G}(x) .
	\end{equation}
\end{itembox}


\begin{proof}
	Let $\bm{F}:\mathbb{R} \rightarrow \mathbb{R}^{m \times k}$
	and $\bm{G}:\mathbb{R} \rightarrow \mathbb{R}^{k \times n}$ .
	For all
	\begin{math}
		i \in \mathbb{N}_m
	\end{math}
	and
	\begin{math}
		j \in \mathbb{N}_n ,
	\end{math}
	\begin{equation*}
		\begin{split}
			(\nabla_x (\bm{F}(x) \bm{G}(x)))_{i,j} &= \nabla_x (\bm{F}(x) \bm{G}(x))_{i,j} \\
			&= \nabla_x \left( \sum_{l=1}^k F_{i,l}(x) G_{l,j}(x) \right) \\
			&= \sum_{l=1}^k ((\nabla_x F_{i,l}(x)) G_{l,j}(x) + F_{i,l}(x) (\nabla_x G_{l,j}(x))) \\
			&= \sum_{l=1}^k (\nabla_x F_{i,l}(x)) G_{l,j}(x) + \sum_{l=1}^k F_{i,l}(x) \nabla_x G_{l,j}(x) \\
			&= \sum_{l=1}^k (\nabla_x \bm{F}(x))_{i,l} G_{l,j}(x) + \sum_{l=1}^k F_{i,l}(x) (\nabla_x \bm{G}(x))_{l,j} \\
			&= ((\nabla_x \bm{F}(x)) \bm{G}(x))_{i,j} + (\bm{F}(x) \nabla_x \bm{G}(x))_{i,j} \\
			&= ((\nabla_x \bm{F}(x)) \bm{G}(x) + \bm{F}(x) \nabla_x \bm{G}(x))_{i,j} .
		\end{split}
	\end{equation*}
	Therefore, $(\ref{ex2155})$ holds.
\end{proof}


\section{Diagonal Matrices}

\subsection{Definition}

\begin{dfn}
	A diagonal matrix is a square matrix whose all the off-diagonal entries are zero.
	Such a matrix with diagonal entries $\bm{d} \coloneq \lbrack d_1 , \ldots , d_n \rbrack^\mathrm{T} \in \mathbb{R}^n$
	is denoted by
	\begin{equation*}
		diag(\bm{d}) \coloneq
		\begin{bmatrix}
			d_1 & 0 & \cdots & 0 \\
			0 & d_2 & \cdots & 0 \\
			\vdots & \vdots & \ddots & \vdots \\
			0 & 0 & \cdots & d_n
		\end{bmatrix}
	\end{equation*}
\end{dfn}


\subsection{Exercise}

\begin{itembox}[l]{Exercise 2.16.3.}
	Let
	\begin{math}
		\bm{x} , \bm{y} \in \mathbb{R}^n .
	\end{math}
	Show that
	\begin{equation}
		\label{ex2163}
		diag(\bm{x}) diag(\bm{y}) = diag(\bm{y}) diag(\bm{x}) .
	\end{equation}
\end{itembox}

\begin{proof}
	Let
	\begin{math}
		\bm{x} , \bm{y} \in \mathbb{R}^n .
	\end{math}
	\begin{equation*}
		\begin{split}
			diag(\bm{x}) diag(\bm{y}) &=
			\begin{bmatrix}
				x_1 &  & \bm{O} \\
				& \ddots & \\
				\bm{O} & & x_n
			\end{bmatrix}
			\begin{bmatrix}
				y_1 &  & \bm{O} \\
				& \ddots & \\
				\bm{O} & & y_n
			\end{bmatrix} \\
			&=
			\begin{bmatrix}
				x_1 y_1 &  & \bm{O} \\
				& \ddots & \\
				\bm{O} & & x_n y_n
			\end{bmatrix} \\
			&=
			\begin{bmatrix}
				y_1 x_1 &  & \bm{O} \\
				& \ddots & \\
				\bm{O} & & y_n x_n
			\end{bmatrix} \\
			&=
			\begin{bmatrix}
				y_1 &  & \bm{O} \\
				& \ddots & \\
				\bm{O} & & y_n
			\end{bmatrix}
			\begin{bmatrix}
				x_1 &  & \bm{O} \\
				& \ddots & \\
				\bm{O} & & x_n
			\end{bmatrix} \\
			&=
			diag(\bm{y}) diag(\bm{x})
		\end{split}
	\end{equation*}
\end{proof}


\section{Orthonormal Matrices}

\subsection{Definition}

\begin{dfn}
	An $m \times n$ matrix $\bm{P}$ is said to be orthonormal if the matrix satisfies
	\begin{math}
		\bm{P}^\mathrm{T} \bm{P} = \bm{I}_n .
	\end{math}
	Symbol
	\begin{math}
		\mathbb{O}^{m \times n}
	\end{math}
	is used to denote the set of $m \times n$ orthonormal matrices.
\end{dfn}

\subsection{Exercise}

\begin{itembox}[l]{Exercise 2.17.2.}
	Let
	\begin{math}
		\bm{P} = \lbrack \bm{p}_1 , \ldots , \bm{p}_n \rbrack \in \mathbb{O}^{m \times n} .
	\end{math}
	Show that
	\begin{math}
		\forall i , \forall j \in \mathbb{N}_n ,
	\end{math}
	\begin{equation}
		\label{ex2172}
		\langle \bm{p}_i , \bm{p}_j \rangle = \delta_{i,j}
	\end{equation}
	where $\delta_{i,j}$ is the Kronecker delta.
\end{itembox}


\begin{proof}
	Let
	\begin{math}
		\bm{P} = \lbrack \bm{p}_1 , \ldots , \bm{p}_n \rbrack \in \mathbb{O}^{m \times n}
	\end{math}
	and
	\begin{math}
		i , j \in \mathbb{N}_n .
	\end{math}
	
	\begin{equation*}
		\begin{split}
			(\bm{P}^{\mathrm{T}} \bm{P})_{i,j} &= \sum_{k=1}^m (\bm{P}^{\mathrm{T}})_{i,k} P_{k,j} \\
			&= \sum_{k=1}^m \bm{P}_{k,i} P_{k,j} \\
			&= \langle \bm{p}_i , \bm{p}_j \rangle
		\end{split}
	\end{equation*}
	Furthermore,
	\begin{equation*}
		\begin{split}
			(\bm{I}_n)_{i,j} &=
			\begin{cases}
				1 & (i=j) \\
				0 & (i \neq j)
			\end{cases} \\
			&= \delta_{i,j}
		\end{split}
	\end{equation*}
	Thus, since $\bm{P}^{\mathrm{T}} \bm{P} = \bm{I}_n$,
	\begin{equation*}
		\langle \bm{p}_i , \bm{p}_j \rangle = (\bm{P}^{\mathrm{T}} \bm{P})_{i,j} = (\bm{I}_n)_{i,j} = \delta_{i,j} .
	\end{equation*}
	Then, we get $(\ref{ex2172})$.
\end{proof}

\begin{itembox}[l]{Exercise 2.17.4.}
	Let
	\begin{math}
		\bm{P} \in \mathbb{O}^{n \times n} .
	\end{math}
	Show that
	\begin{equation}
		\label{ex2174}
		\bm{P}^{\mathrm{T}} = \bm{P}^{-1} .
	\end{equation}
\end{itembox}


\begin{proof}
	Let
	\begin{math}
		\bm{P} \in \mathbb{O}^{n \times n} .
	\end{math}
	Since
	\begin{math}
		\bm{P} \in \mathbb{O}^{n \times n},
	\end{math}
	\begin{equation*}
		\bm{P}^{\mathrm{T}} \bm{P} = \bm{I}_n .
	\end{equation*}
	Recall the definition of inverse of square matrices. Then, we get
	\begin{equation*}
		\bm{P}^{-1} = \bm{P}^{\mathrm{T}} .
	\end{equation*}
	Thus, $(\ref{ex2174})$ holds.
\end{proof}


\begin{itembox}[l]{Exercise 2.17.6.}
	Let
	\begin{math}
		\bm{P} \in \mathbb{O}^{ n \times n} .
	\end{math}
	Show that
	\begin{equation}
		\label{ex2176}
		det ({\bm{P}}) \in \{ \pm 1 \} .
	\end{equation}
\end{itembox}


\begin{proof}
	Let
	\begin{math}
		\bm{P} \in \mathbb{O}^{ n \times n} .
	\end{math}
	Recall that $det(\bm{AB}) = det(\bm{A}) det(\bm{B})$ and
	$det(\bm{A}^{\mathrm{T}}) = det(\bm{A})$ for $\bm{A} , \bm{B} \in \mathbb{R}^{n \times n}$.
	Then, 
	\begin{equation*}
		det(\bm{P}^{\mathrm{T}} \bm{P}) = det(\bm{P}^{\mathrm{T}}) det(\bm{P}) = det(\bm{P}) det(\bm{P}) = (det(\bm{P}))^2 .
	\end{equation*}
	Recall that
	\begin{equation*}
		det(\bm{I}_n) = 1 .
	\end{equation*}
	Then, from the definition of orthonormal matrices,
	\begin{equation*}
		(det(\bm{P}))^2 = det(\bm{P}^{\mathrm{T}} \bm{P}) = det(\bm{I}_n) = 1 .
	\end{equation*}
	By solving this equation, we have
	\begin{math}
		det(\bm{P}) = \pm 1 .
	\end{math}
	Hence, $det(\bm{P}) \in \{\pm 1\}$ .
\end{proof}


\begin{itembox}[l]{Exercise 2.17.8.}
	Let
	\begin{math}
		\bm{P} \in \mathbb{O}^{m \times k}
	\end{math}
	and
	\begin{math}
		\bm{Q} \in \mathbb{O}^{k \times n}
	\end{math}
	where $m \geq k \geq n$. Show that $\bm{PQ} \in \mathbb{O}^{m \times n}$.
	
	
\end{itembox}


\begin{proof}
	Let
	\begin{math}
		\bm{P} \in \mathbb{O}^{m \times k}
	\end{math}
	and
	\begin{math}
		\bm{Q} \in \mathbb{O}^{k \times n}
	\end{math}
	where $m \geq k \geq n$.
	Since $\bm{P} \in \mathbb{O}^{m \times k}$,
	\begin{equation*}
		\bm{P}^{\mathrm{T}} \bm{P} = \bm{I}_k .
	\end{equation*}
	Similarly,
	\begin{equation*}
		\bm{Q}^{\mathrm{T}} \bm{Q} = \bm{I}_n .
	\end{equation*}
	Then,
	\begin{equation*}
		\begin{split}
			(\bm{PQ})^{\mathrm{T}} \bm{PQ} &= \bm{Q}^{\mathrm{T}} \bm{P}^{\mathrm{T}} \bm{PQ} \\
			&= \bm{Q}^{\mathrm{T}} (\bm{P}^{\mathrm{T}} \bm{P}) \bm{Q} \\
			&= \bm{Q}^{\mathrm{T}} \bm{I}_k \bm{Q} \\
			&= \bm{Q}^{\mathrm{T}} \bm{Q} \\
			&= \bm{I}_n .
		\end{split}
	\end{equation*}
	Hence, $\bm{PQ} \in \mathbb{O}^{m \times n}$.
\end{proof}

\end{document}
