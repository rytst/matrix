\documentclass{article}
\usepackage{amsmath}
\usepackage{amsthm}
\usepackage{amsfonts}
\usepackage{mathtools}
\usepackage{mathrsfs}
\usepackage{ascmac}
\usepackage{bm}
\usepackage[margin=13truemm]{geometry}
\theoremstyle{plain}
\newtheorem{dfn}{Definition}[subsection]
\title{Matrix Algebra Marathon}
\author{J2200071 Ryuto Saito}
\date{\today}

\begin{document}
\maketitle

\section{Basic Matrix Identities}

\subsection{Definition}

\begin{dfn}
	The transpose of an $m \times n$ matrix $\bm{A}$, to be denoted by $A^\mathrm{T}$,
	is the $n \times m$ matrix whose $(j , i)$-th entry is $(i , j)$-th  entry of $\bm{A}$. Namely,
	\begin{equation}
		\left(
		\begin{bmatrix}
			A_{1,1} & \cdots & A_{1,n} \\
			\vdots & \ddots & \vdots \\
			A_{m,1} & \cdots & A_{m,n}
		\end{bmatrix}
		\right)^\mathrm{T}
		=
		\begin{bmatrix}
			A_{1,1} & \cdots & A_{m,1} \\
			\vdots & \ddots & \vdots \\
			A_{1,n} & \cdots & A_{m,n}
		\end{bmatrix}
	\end{equation}
\end{dfn}

\begin{dfn}
	The inverse of a square $n \times n$ matrix $\bm{A}$, to be denoted by $\bm{A}^{-1}$, is the $n \times n$ matrix such that
	\begin{equation}
		\bm{A}^{-1} \bm{A} = \bm{I}_n .
	\end{equation}
	The matrix $\bm{A}$ is said to be non-singlar if $\bm{A}^{-1}$ exists.
\end{dfn}


\subsection{Exercise}

\begin{itembox}[l]{Exercise 2.5.2.}
	Let
	\begin{math}
		\bm{A} \in \mathbb{R}^{m \times p} , \bm{B} , \bm{C} \in \mathbb{R}^{p \times n} .
	\end{math}
	Show that
	\begin{math}
		\bm{A} (\bm{B} + \bm{C}) = \bm{AB} + \bm{AC} .
	\end{math}

\end{itembox}

\begin{proof}
	Let
	\begin{math}
		\bm{A} \in \mathbb{R}^{m \times p} , \bm{B} , \bm{C} \in \mathbb{R}^{p \times n} .
	\end{math}
	For all
	\begin{math}
		i \in \mathbb{N}_m ,
	\end{math}
	and
	\begin{math}
		j \in \mathbb{N}_n ,
	\end{math}
	\begin{equation*}
		\begin{split}
			\left( \bm{A} (\bm{B} + \bm{C}) \right)_{i,j} &= \sum_{k=1}^p \bm{A}_{i,k} (\bm{B} + \bm{C})_{k,j} \\
			&= \sum_{k=1}^p \bm{A} (\bm{B}_{k,j} + \bm{C}_{k,j}) \\
			&= \sum_{k=1}^p (\bm{A}_{i,k} \bm{B}_{k,j} + \bm{A}_{i,k} \bm{C}_{k,j}) \\
			&= \sum_{k=1}^p \bm{A}_{i,k} \bm{B}_{k,j} + \sum_{k=1}^p \bm{A}_{i,k} \bm{C}_{k,j} \\
			&= \bm{AB}_{i,j} + \bm{AC}_{i,j} \\
			&= (\bm{AB} + \bm{AC})_{i,j} .
		\end{split}
	\end{equation*}
	From this $\bm{A} (\bm{B} + \bm{C}) = \bm{AB} + \bm{AC}$ follows.
\end{proof}

\begin{itembox}[l]{Exercise 2.5.5.}
	Let
	\begin{math}
		\bm{A} , \bm{B} \in \mathbb{R}^{m \times n} .
	\end{math}
	Show that
	\begin{math}
		(\bm{A} + \bm{B})^\mathrm{T} = \bm{A}^\mathrm{T} + \bm{B}^\mathrm{T} .
	\end{math}
\end{itembox}

\begin{proof}
	Let
	\begin{math}
		\bm{A} , \bm{B} \in \mathbb{R}^{m \times n}.
	\end{math}
	For all
	\begin{math}
		i \in \mathbb{N}_m ,
	\end{math}
	and
	\begin{math}
		j \in \mathbb{N}_n ,
	\end{math}
	\begin{equation*}
		\begin{split}
			\left((\bm{A} + \bm{B})^\mathrm{T} \right)_{i,j} &= (\bm{A} + \bm{B})_{j,i} \\
			&= \bm{A}_{j,i} + \bm{B}_{j,i} \\
			&= \bm{A}_{i,j}^ \mathrm{T} + \bm{B}_{i,j}^\mathrm{T} \\
			&= (\bm{A}^ \mathrm{T} + \bm{B}^ \mathrm{T})_{i,j} .
		\end{split}
	\end{equation*}
	From this $(\bm{A} + \bm{B})^\mathrm{T} = \bm{A}^\mathrm{T} + \bm{B}^\mathrm{T}$ follows.
\end{proof}

\begin{itembox}[l]{Exercise 2.5.7.}
	Let
	\begin{math}
		\bm{A} , \bm{B} \in \mathbb{R}^{m \times p} , \bm{C} \in \mathbb{R}^{p \times n} .
	\end{math}
	Show that
	\begin{math}
		(\bm{A} + \bm{B}) \bm{C} = \bm{AC} + \bm{BC} .
	\end{math}
\end{itembox}

\begin{proof}
	Let
	\begin{math}
		\bm{A} , \bm{B} \in \mathbb{R}^{m \times p} , \bm{C} \in \mathbb{R}^{p \times n} .
	\end{math}
	For all
	\begin{math}
		i \in \mathbb{N}_m ,
	\end{math}
	and
	\begin{math}
		j \in \mathbb{N}_n ,
	\end{math}
	\begin{equation*}
		\begin{split}
			((\bm{A} + \bm{B}) \bm{C})_{i,j} &= \sum_{k=1}^p (\bm{A} + \bm{B})_{i,k} \bm{C}_{k,j} \\
			&= \sum_{k=1}^p (\bm{A}_{i,k} + \bm{B}_{i,k}) \bm{C}_{k,j} \\
			&= \sum_{k=1}^p (\bm{A}_{i,k} \bm{C}_{k,j} + \bm{B}_{i,k} \bm{C}_{k,j}) \\
			&= \sum_{k=1}^p (\bm{A}_{i,k} \bm{C}_{k,j}) + \sum_{k=1}^p (\bm{B}_{i,k} \bm{C}_{k,j}) \\
			&= (\bm{A} \bm{C})_{i,j} + (\bm{B} \bm{C})_{i,j} \\
			&= (\bm{A} \bm{C} + \bm{B} \bm{C})_{i,j} .
		\end{split}
	\end{equation*}
	From this $(\bm{A} + \bm{B}) \bm{C} = \bm{AC} + \bm{BC}$ follows.
\end{proof}

\begin{itembox}[l]{Exercise 2.5.9.}
	Let
	\begin{math}
		\bm{A} \in \mathbb{R}^{m \times n} , \bm{B} \in \mathbb{R}^{n \times p} ,
	\end{math}
	and
	\begin{math}
		\bm{C} \in \mathbb{R}^{n \times q} .
	\end{math}
	Show that
	\begin{equation}
		\bm{A} \lbrack \bm{B} , \bm{C} \rbrack = \lbrack \bm{AB} , \bm{AC} \rbrack .
	\end{equation}
\end{itembox}

\begin{proof}
	Let
	\begin{math}
		\bm{A} \in \mathbb{R}^{m \times n} , \bm{B} \in \mathbb{R}^{n \times p} ,
	\end{math}
	and
	\begin{math}
		\bm{C} \in \mathbb{R}^{n \times q} .
	\end{math}
	\begin{equation*}
		\begin{split}
			\bm{A} \lbrack \bm{B} , \bm{C} \rbrack &=
			\begin{bmatrix}
				A_{1,1} & \cdots & A_{1,n} \\
				\vdots & \ddots & \vdots \\
				A_{m,1} & \cdots & A_{m,n}
			\end{bmatrix}
			\begin{bmatrix}
				B_{1,1} & \cdots & B_{1,p} & C_{1,1} & \cdots & C_{1,q} \\
				\vdots & \ddots & \vdots & \vdots & \ddots & \vdots \\
				B_{n,1} & \cdots & B_{n,p} & C_{n,1} & \cdots & C_{n,q}
			\end{bmatrix} \\
			&=
			\begin{bmatrix}
				\sum_{i=1}^n A_{1,i} B_{i,1} & \cdots & \sum_{i=1}^n A_{1,i} B_{i,p} & \sum_{i=1}^n A_{1,i} C_{i,1} & \cdots & \sum_{i=1}^n A_{1,i} C_{i,q} \\
				\vdots & \ddots & \vdots & \vdots & \ddots & \vdots \\
				\sum_{i=1}^n A_{m,i} B_{i,1} & \cdots & \sum_{i=1}^n A_{m,i} B_{i,p} & \sum_{i=1}^n A_{m,i} C_{i,1} & \cdots & \sum_{i=1}^n A_{m,i} C_{i,q}
			\end{bmatrix} \\
			&= \lbrack \bm{AB} , \bm{AC} \rbrack
		\end{split}
	\end{equation*}
\end{proof}

\begin{itembox}[l]{Exercise 2.5.11.}
	Let
	\begin{math}
		\bm{A} \in \mathbb{R}^{m \times p} , \bm{B} \in \mathbb{R}^{m \times q} , \bm{C} \in \mathbb{R}^{p \times n} ,
	\end{math}
	and
	\begin{math}
		\bm{D} \in \mathbb{R}^{q \times n} .
	\end{math}
	Show that
	\begin{equation}
		\lbrack \bm{A} , \bm{B} \rbrack 
		\begin{bmatrix}
			\bm{C} \\
			\bm{D}
		\end{bmatrix}
		= \bm{AC} + \bm{BD} .
	\end{equation}
\end{itembox}

\begin{proof}
	Let
	\begin{math}
		\bm{A} \in \mathbb{R}^{m \times p} , \bm{B} \in \mathbb{R}^{m \times q} , \bm{C} \in \mathbb{R}^{p \times n} ,
	\end{math}
	and
	\begin{math}
		\bm{D} \in \mathbb{R}^{q \times n} .
	\end{math}
	\begin{equation*}
		\begin{split}
			\lbrack \bm{A} , \bm{B} \rbrack
			\begin{bmatrix}
				\bm{C} \\
				\bm{D}
			\end{bmatrix}
			&=
			\begin{bmatrix}
				A_{1,1} & \cdots & A_{1,p} & B_{1,1} & \cdots & B_{1,q} \\
				\vdots & \ddots & \vdots & \vdots & \ddots & \vdots \\
				A_{m,1} & \cdots & A_{m,p} & B_{m,1} & \cdots & B_{m,q}
			\end{bmatrix}
			\begin{bmatrix}
				C_{1,1} & \cdots & C_{1,n} \\
				\vdots & \ddots & \vdots \\
				C_{p.1} & \cdots & C_{p,n} \\
				D_{1.1} & \cdots & D_{1,n} \\
				\vdots & \ddots & \vdots \\
				D_{q,1} & \cdots & D_{q,n}
			\end{bmatrix} \\
			&=
			\begin{bmatrix}
				\sum_{i=1}^p A_{1,i} C_{i,1} + \sum_{i=1}^q B_{1,i} D_{i,1} & \cdots & \sum_{i=1}^p A_{1,i} C_{i,n} + \sum_{i=1}^q B_{1,i} D_{i,n} \\
				\vdots & \ddots & \vdots \\
				\sum_{i=1}^p A_{m,i} C_{i,1} + \sum_{i=1}^q B_{m,i} D_{i,1} & \cdots & \sum_{i=1}^p A_{m,i} C_{i,n} + \sum_{i=1}^q B_{m,i} D_{i,n}
			\end{bmatrix} \\
			&=
			\begin{bmatrix}
				\sum_{i=1}^p A_{1,i} C_{i,1} & \cdots & \sum_{i=1}^p A_{1,i} C_{i,n} \\
				\vdots & \ddots & \vdots \\
				\sum_{i=1}^p A_{m,i} C_{i,1} & \cdots & \sum_{i=1}^p A_{m,i} C_{i,n}
			\end{bmatrix}
			+
			\begin{bmatrix}
				\sum_{i=1}^q B_{1,i} D_{i,1} & \cdots & \sum_{i=1}^q B_{1,i} D_{i,n} \\
				\vdots & \ddots & \vdots \\
				\sum_{i=1}^q B_{m,i} D_{i,1} & \cdots & \sum_{i=1}^q B_{m,i} D_{i,n}
			\end{bmatrix} \\
			&=
			\bm{AC} + \bm{BD}
		\end{split}
	\end{equation*}
\end{proof}

\begin{itembox}[l]{Exercise 2.5.13.}
	Let
	\begin{math}
		\bm{A}^{(1)} \in \mathbb{R}^{m \times n_1} , \bm{A}^{(2)} \in \mathbb{R}^{m \times n_2} ,
		\bm{B}^{(1,1)} \in \mathbb{R}^{n_1 \times p_1} , \bm{B}^{(1,2)} \in \mathbb{R}^{n_1 \times p_2} ,
        \bm{B}^{(2,1)} \in \mathbb{R}^{n_2 \times p_1} ,
	\end{math}
	and
	\begin{math}
		\bm{B}^{(2,2)} \in \mathbb{R}^{n_2 \times p_2} .
	\end{math}
	Show that
	\begin{equation}
		\begin{split}
			&\lbrack \bm{A}^{(1)} , \bm{A}^{(2)} \rbrack
		    \begin{bmatrix}
			    \bm{B}^{(1,1)} & \bm{B}^{(1,2)} \\
			    \bm{B}^{(2,1)} & \bm{B}^{(2,2)}
		    \end{bmatrix} \\
			&=
		    \lbrack \bm{A}^{(1)} \bm{B}^{(1,1)} + \bm{A}^{(2)} \bm{B}^{(2,1)} , \bm{A}^{(1)} \bm{B}^{(1,2)} + \bm{A}^{(2)} \bm{B}^{(2,2)} \rbrack .
		\end{split}
	\end{equation}
\end{itembox}

\begin{proof}
	Let
	\begin{math}
		\bm{A}^{(1)} \in \mathbb{R}^{m \times n_1} , \bm{A}^{(2)} \in \mathbb{R}^{m \times n_2} ,
		\bm{B}^{(1,1)} \in \mathbb{R}^{n_1 \times p_1} , \bm{B}^{(1,2)} \in \mathbb{R}^{n_1 \times p_2} ,
        \bm{B}^{(2,1)} \in \mathbb{R}^{n_2 \times p_1} ,
	\end{math}
	and
	\begin{math}
		\bm{B}^{(2,2)} \in \mathbb{R}^{n_2 \times p_2} .
	\end{math}
	\footnotesize
	\begin{equation*}
		\begin{split}
			&\lbrack \bm{A}^{(1)} , \bm{A}^{(2)} \rbrack
		    \begin{bmatrix}
			    \bm{B}^{(1,1)} & \bm{B}^{(1,2)} \\
			    \bm{B}^{(2,1)} & \bm{B}^{(2,2)}
		    \end{bmatrix} \\
			&=
			\begin{bmatrix}
				A_{1,1}^{(1)} & \cdots & A_{1,n_1}^{(1)} & A_{1,1}^{(2)} & \cdots & A_{1,n_2}^{(2)} \\
				\vdots & \ddots & \vdots & \vdots & \ddots & \vdots \\
				A_{m,1}^{(1)} & \cdots & A_{m,n_1}^{(1)} & A_{m,1}^{(2)} & \cdots & A_{m,n_2}^{(2)}
			\end{bmatrix}
			\begin{bmatrix}
				B_{1,1}^{(1,1)} & \cdots & B_{1,p_1}^{(1,1)} & B_{1,1}^{(1,2)} & \cdots & B_{1,p_2}^{(1,2)} \\
				\vdots & \ddots & \vdots & \vdots & \ddots & \vdots \\
				B_{n_1,1}^{(1,1)} & \cdots & B_{n_1,p_1}^{(1,1)} & B_{n_1,1}^{(1,2)} & \cdots & B_{n_1,p_2}^{(1,2)} \\
				B_{1,1}^{(2,1)} & \cdots & B_{1,p_1}^{(2,1)} & B_{1,1}^{(2,2)} & \cdots & B_{1,p_2}^{(2,2)} \\
				\vdots & \ddots & \vdots & \vdots & \ddots & \vdots \\
				B_{n_2,1}^{(2,1)} & \cdots & B_{n_2,p_1}^{(2,1)} & B_{n_2,1}^{(2,2)} & \cdots & B_{n_2,p_2}^{(2,2)}
			\end{bmatrix} \\
			&=
			\begin{bmatrix}
				\sum\limits_{i=1}^{n_1} A_{1,i}^{(1)} B_{i,1}^{(1,1)} + \sum\limits_{i=1}^{n_2} A_{1,i}^{(2)} B_{i,1}^{(2,1)} &
				\cdots &
				\sum\limits_{i=1}^{n_1} A_{1,i}^{(1)} B_{i,p_1}^{(1,1)} + \sum\limits_{i=1}^{n_2} A_{1,i}^{(2)} B_{i,p_1}^{(2,1)} &
				\sum\limits_{i=1}^{n_1} A_{1,i}^{(1)} B_{i,1}^{(1,2)} + \sum\limits_{i=1}^{n_2} A_{1,i}^{(2)} B_{i,1}^{(2,2)} &
				\cdots &
				\sum\limits_{i=1}^{n_1} A_{1,i}^{(1)} B_{i,p_2}^{(1,2)} + \sum\limits_{i=1}^{n_2} A_{1,i}^{(2)} B_{i,p_2}^{(2,2)} \\
				\vdots & \ddots & \vdots & \vdots & \ddots & \vdots \\
				\sum\limits_{i=1}^{n_1} A_{m,i}^{(1)} B_{i,1}^{(1,1)} + \sum\limits_{i=1}^{n_2} A_{m,i}^{(2)} B_{i,1}^{(2,1)} &
				\cdots &
				\sum\limits_{i=1}^{n_1} A_{m,i}^{(1)} B_{i,p_1}^{(1,1)} + \sum\limits_{i=1}^{n_2} A_{m,i}^{(2)} B_{i,p_1}^{(2,1)} &
				\sum\limits_{i=1}^{n_1} A_{m,i}^{(1)} B_{i,1}^{(1,2)} + \sum\limits_{i=1}^{n_2} A_{m,i}^{(2)} B_{i,1}^{(2,2)} &
				\cdots &
				\sum\limits_{i=1}^{n_1} A_{m,i}^{(1)} B_{i,p_2}^{(1,2)} + \sum\limits_{i=1}^{n_2} A_{m,i}^{(2)} B_{i,p_2}^{(2,2)} \\
			\end{bmatrix} \\
			&=
			\begin{bmatrix}
				\sum\limits_{i=1}^{n_1} A_{1,i}^{(1)} B_{i,1}^{(1,1)} &
				\cdots &
				\sum\limits_{i=1}^{n_1} A_{1,i}^{(1)} B_{i,p_1}^{(1,1)} &
				\sum\limits_{i=1}^{n_1} A_{1,i}^{(1)} B_{i,1}^{(1,2)} &
				\cdots &
				\sum\limits_{i=1}^{n_1} A_{1,i}^{(1)} B_{i,p_2}^{(1,2)} \\
				\vdots & \ddots & \vdots & \vdots & \ddots & \vdots \\
				\sum\limits_{i=1}^{n_1} A_{m,i}^{(1)} B_{i,1}^{(1,1)} &
				\cdots &
				\sum\limits_{i=1}^{n_1} A_{m,i}^{(1)} B_{i,p_1}^{(1,1)} &
				\sum\limits_{i=1}^{n_1} A_{m,i}^{(1)} B_{i,1}^{(1,2)} &
				\cdots &
				\sum\limits_{i=1}^{n_1} A_{m,i}^{(1)} B_{i,p_2}^{(1,2)} \\
			\end{bmatrix} \\
			&+
			\begin{bmatrix}
				\sum\limits_{i=1}^{n_2} A_{1,i}^{(2)} B_{i,1}^{(2,1)} &
				\cdots &
				\sum\limits_{i=1}^{n_2} A_{1,i}^{(2)} B_{i,p_1}^{(2,1)} &
				\sum\limits_{i=1}^{n_2} A_{1,i}^{(2)} B_{i,1}^{(2,2)} &
				\cdots &
				\sum\limits_{i=1}^{n_2} A_{1,i}^{(2)} B_{i,p_2}^{(2,2)} \\
				\vdots & \ddots & \vdots & \vdots & \ddots & \vdots \\
				\sum\limits_{i=1}^{n_2} A_{m,i}^{(2)} B_{i,1}^{(2,1)} &
				\cdots &
				\sum\limits_{i=1}^{n_2} A_{m,i}^{(2)} B_{i,p_1}^{(2,1)} &
				\sum\limits_{i=1}^{n_2} A_{m,i}^{(2)} B_{i,1}^{(2,2)} &
				\cdots &
				\sum\limits_{i=1}^{n_2} A_{m,i}^{(2)} B_{i,p_2}^{(2,2)} \\
			\end{bmatrix} \\
			&= \lbrack \bm{A}^{(1)} \bm{B}^{(1,1)} , \bm{A}^{(1)} \bm{B}^{(1,2)} \rbrack + \lbrack \bm{A}^{(2)} \bm{B}^{(2,1)} , \bm{A}^{(2)} \bm{B}^{(2,2)} \rbrack \\
			&= \lbrack \bm{A}^{(1)} \bm{B}^{(1,1)} + \bm{A}^{(2)} \bm{B}^{(2,1)} , \bm{A}^{(1)} \bm{B}^{(1,2)} + \bm{A}^{(2)} \bm{B}^{(2,2)} \rbrack
		\end{split}
	\end{equation*}
	\normalsize
\end{proof}

\begin{itembox}[l]{Exercise 2.5.19.}
	Show that
	\begin{math}
		(\bm{ABC})^{-1} = \bm{C}^{-1} \bm{B}^{-1} \bm{A}^{-1}
	\end{math}
	where
	\begin{math}
		\bm{A} , \bm{B} , \bm{C} \in \mathbb{R}^{n \times n}
	\end{math}
	are non-singular.
\end{itembox}

\begin{proof}
	Let
	\begin{math}
		\bm{A} , \bm{B} , \bm{C} \in \mathbb{R}^{n \times n}
	\end{math}
	be non-singular.
	\begin{equation*}
		\begin{split}
			(\bm{C}^{-1} \bm{B}^{-1} \bm{A}^{-1}) (\bm{ABC}) &= \bm{C}^{-1} \bm{B}^{-1} \bm{A}^{-1} \bm{ABC} \\
			&= \bm{C}^{-1} \bm{B}^{-1} (\bm{A}^{-1} \bm{A}) \bm{BC} \\
			&= \bm{C}^{-1} \bm{B}^{-1} \bm{I}_n \bm{BC} \\
			&= \bm{C}^{-1} \bm{B}^{-1} \bm{BC} \\
			&= \bm{C}^{-1} (\bm{B}^{-1} \bm{B}) \bm{C} \\
			&= \bm{C}^{-1} \bm{I}_n \bm{C} \\
			&= \bm{C}^{-1} \bm{C} \\
			&= \bm{I}_n
		\end{split}
	\end{equation*}
	Then, by definition 2.5.16. , we get
	\begin{math}
		(\bm{ABC})^{-1} = \bm{C}^{-1} \bm{B}^{-1} \bm{A}^{-1} .
	\end{math}
\end{proof}

\begin{itembox}[l]{Exercise 2.5.22.}
	Show the Woodbury formula
	\begin{equation}
		\label{woodbury}
		(\bm{A} + \bm{B} \bm{D}^{-1} \bm{C})^{-1} = \bm{A}^{-1} - \bm{A}^{-1} \bm{B} (\bm{D} + \bm{C} \bm{A}^{-1} \bm{B})^{-1} \bm{C} \bm{A}^{-1} .
	\end{equation}
	where
	\begin{math}
		\bm{A} , \bm{B} , \bm{C},
	\end{math}
	and $\bm{D}$ are matrices with the correct size.
\end{itembox}

\begin{proof}
	Let
	\begin{math}
		\bm{A} \in \mathbb{R}^{n \times n} , \bm{B} , \bm{C} ,
	\end{math}
	and, $\bm{D}$ be matrices with the correct size.
	\begin{equation*}
		\begin{split}
			&(\bm{A}^{-1} - \bm{A}^{-1} \bm{B} (\bm{D} + \bm{C} \bm{A}^{-1} \bm{B})^{-1} \bm{C} \bm{A}^{-1})(\bm{A} + \bm{B} \bm{D}^{-1} \bm{C}) \\
			=& (\bm{A}^{-1} - \bm{A}^{-1} \bm{B} (\bm{D} + \bm{C} \bm{A}^{-1} \bm{B})^{-1} \bm{C} \bm{A}^{-1}) \bm{A} \\
			&+ (\bm{A}^{-1} - \bm{A}^{-1} \bm{B} (\bm{D} + \bm{C} \bm{A}^{-1} \bm{B})^{-1} \bm{C} \bm{A}^{-1}) \bm{B} \bm{D}^{-1} \bm{C} \\
			=& \bm{I}_n - \bm{A}^{-1} \bm{B} (\bm{D} + \bm{C} \bm{A}^{-1} \bm{B})^{-1} \bm{C} \\
			&+ \bm{A}^{-1} \bm{B} \bm{D}^{-1} \bm{C} - \bm{A}^{-1} \bm{B} (\bm{D} + \bm{C} \bm{A}^{-1} \bm{B})^{-1} \bm{C} \bm{A}^{-1} \bm{B} \bm{D}^{-1} \bm{C} \\
			=& \bm{I}_n + \bm{A}^{-1} \bm{B} \bm{D}^{-1} \bm{C} - \bm{A}^{-1} \bm{B} (\bm{D} + \bm{C} \bm{A}^{-1} \bm{B})^{-1} (\bm{D} \bm{D}^{-1} + \bm{C} \bm{A}^{-1} \bm{B} \bm{D}^{-1}) \bm{C} \\
			=& \bm{I}_n + \bm{A}^{-1} \bm{B} \bm{D}^{-1} \bm{C} - \bm{A}^{-1} \bm{B} (\bm{D} + \bm{C} \bm{A}^{-1} \bm{B})^{-1} (\bm{D} + \bm{C} \bm{A}^{-1} \bm{B}) \bm{D}^{-1} \bm{C} \\
			=& \bm{I}_n + \bm{A}^{-1} \bm{B} \bm{D}^{-1} \bm{C} - \bm{A}^{-1} \bm{B} \bm{D}^{-1} \bm{C} \\
			=& \bm{I}_n
		\end{split}
	\end{equation*}
	Then, by definition 2.5.16. , we get $(\ref{woodbury})$.
\end{proof}

\begin{itembox}[l]{Exercise 2.5.25.}
	Let
	\begin{math}
		\bm{A} = \lbrack \bm{a}_1 , \ldots , \bm{a}_m \rbrack^\mathrm{T} \in \mathbb{R}^{m \times n} .
	\end{math}
	Note that $i$-th row of $\bm{A}$ is $\bm{a}_i^\mathrm{T}$. Show that,
	\begin{math}
		\forall k \in \mathbb{N}_m,
	\end{math}
	\begin{equation}
		\label{ex2525}
		\bm{e}_k^\mathrm{T} \bm{A} = \bm{a}_k^\mathrm{T}
	\end{equation}
	where $\bm{e}_k$ is a unit vector with $k$-th entry one and the other entries zero.
\end{itembox}

\begin{proof}
	Let
	\begin{math}
		\bm{A} = \lbrack \bm{a}_1 , \ldots , \bm{a}_m \rbrack^\mathrm{T} \in \mathbb{R}^{m \times n}
	\end{math}
	and
	\begin{math}
		k \in \mathbb{N}_m .
	\end{math}
	\begin{equation*}
		\begin{split}
			\bm{e}_k^\mathrm{T} \bm{A} &=
			\bm{e}_k^\mathrm{T}
			\begin{bmatrix}
				A_{1,1} & \cdots & A_{1,n} \\
				\vdots & \ddots & \vdots \\
				A_{m,1} & \cdots & A_{m,n}
			\end{bmatrix} \\
			&= \lbrack A_{k,1} , \ldots , A_{k,n} \rbrack \\
		\end{split}
	\end{equation*}
	Since $i$-th row of $\bm{A}$ is $\bm{a}^\mathrm{T}$,
	\begin{equation*}
		\lbrack A_{k,1} , \ldots , A_{k,n} \rbrack = \bm{a}_k^\mathrm{T} .
	\end{equation*}
	Thus, $(\ref{ex2525})$ holds.
\end{proof}

\begin{itembox}[l]{Exercise 2.5.27.}
	Let
	\begin{math}
		\bm{A} = \lbrack \bm{a}_1 , \ldots , \bm{a}_m \rbrack \in \mathbb{R}^{n \times m} .
	\end{math}
	Note that $i$-th column of $\bm{A}$ is $\bm{a}_i$. Let $\bm{x} \in \mathbb{R}^n$ and $k \in \mathbb{N}_m$. Show that
	\begin{equation}
		\langle \bm{e}_k , \bm{A}^\mathrm{T} \bm{x} \rangle = \langle \bm{a}_k , \bm{x} \rangle
	\end{equation}
	where $\bm{e}_k$ is a unit vector with $k$-th entry one and the other entries zero.
\end{itembox}

\begin{proof}
	Let
	\begin{math}
		\bm{A} = \lbrack \bm{a}_1 , \ldots , \bm{a}_m \rbrack \in \mathbb{R}^{n \times m} ,
	\end{math}
	\begin{math}
		\bm{x} \in \mathbb{R}^n
	\end{math}
	and
	\begin{math}
		k \in \mathbb{N}_m .
	\end{math}
	\begin{equation*}
		\begin{split}
			\bm{A}^\mathrm{T} \bm{x} &=
			\begin{bmatrix}
				A_{1,1} & \cdots & A_{n,1} \\
				\vdots & \ddots & \vdots \\
				A_{1,m} & \cdots & A_{n,m}
			\end{bmatrix}
			\begin{bmatrix}
				x_1 \\
				\vdots \\
				x_n
			\end{bmatrix} \\
			&=
			\begin{bmatrix}
				\bm{a}_1^\mathrm{T} \bm{x} \\
				\vdots \\
				\bm{a}_m^\mathrm{T} \bm{x}
			\end{bmatrix}
		\end{split}
	\end{equation*}
	Thus,
	\begin{equation*}
		\begin{split}
			\langle \bm{e}_k , \bm{A}^\mathrm{T} \bm{x} \rangle 
			&= \bm{a}_k^\mathrm{T} \bm{x} \\
			&= \langle \bm{a}_k , \bm{x} \rangle .
		\end{split}
	\end{equation*}
\end{proof}

\end{document}
