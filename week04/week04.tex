\documentclass{article}
\usepackage{amsmath}
\usepackage{amsthm}
\usepackage{amsfonts}
\usepackage{mathtools}
\usepackage{mathrsfs}
\usepackage{ascmac}
\usepackage{bm}
\theoremstyle{plain}
\newtheorem{dfn}{Definition}[subsection]
\title{Matrix Algebra Marathon}
\author{J2200071 Ryuto Saito}
\date{\today}

\begin{document}
\maketitle

\section{Derivatives of Function of Vectors}

\subsection{Exercise}

\begin{itembox}[l]{Exercise 2.12.2.}
	If we define a function $f$: $\mathbb{R}^2 \rightarrow \mathbb{R}$ as $f(\bm{x}) = 3 x_1 - x_2$,
	show that the derivative is given by
	\begin{equation}
		\nabla_{\bm{x}} f(\bm{x}) = \lbrack 3 , -1 \rbrack^{\mathrm{T}} .
	\end{equation}
\end{itembox}

\begin{proof}
	Let
	\begin{math}
		\bm{x} \in \mathbb{R}^2 .
	\end{math}
	
	\begin{equation*}
		\nabla_{\bm{x}} f(\bm{x}) =
		\begin{bmatrix}
			\nabla_{x_1} f(\bm{x}) \\
			\nabla_{x_2} f(\bm{x})
		\end{bmatrix}
		=
		\begin{bmatrix}
			\nabla_{x_1} (3 x_1 - x_2) \\
			\nabla_{x_2} (3 x_1 - x_2)
		\end{bmatrix}
		=
		\begin{bmatrix}
			3 \\
			-1
		\end{bmatrix}
		=
		\lbrack 3 , -1 \rbrack^{\mathrm{T}}
	\end{equation*}
\end{proof}

\begin{itembox}[l]{Exercise 2.12.4.}
	Let a function $f$: $\mathbb{R}^n \rightarrow \mathbb{R}$ be defined as
	\begin{equation}
		\label{f2124}
		f(\bm{x}) \coloneq \sum_{i=1}^n \sin(x_i) .
	\end{equation}
	Show that the derivative is given by
	\begin{equation}
		\nabla_{\bm{x}} f(\bm{x}) = \lbrack \cos(x_1) , \ldots , \cos(x_n) \rbrack^\mathrm{T} .
	\end{equation}
\end{itembox}


\begin{proof}
	Let
	\begin{math}
		\bm{x} \in \mathbb{R}^n .
	\end{math}
	
	\begin{equation*}
		\begin{split}
			\nabla_{\bm{x}} f(\bm{x}) &=
			\begin{bmatrix}
				\nabla_{x_1} f(\bm{x}) \\
				\vdots \\
				\nabla_{x_n} f(\bm{x})
			\end{bmatrix} \\
			&=
			\begin{bmatrix}
				\nabla_{x_1} (\sum_{i=1}^n \sin(x_i)) \\
				\vdots \\
				\nabla_{x_n} (\sum_{i=1}^n \sin(x_i))
			\end{bmatrix} \\
			&=
			\begin{bmatrix}
				\nabla_{x_1} \sin(x_1) \\
				\vdots \\
				\nabla_{x_n} \sin(x_n)
			\end{bmatrix} \\
			&=
			\begin{bmatrix}
				\cos(x_1) \\
				\vdots \\
				\cos(x_n)
			\end{bmatrix} \\
			&=
			\lbrack \cos(x_1) , \ldots , \cos(x_n) \rbrack^\mathrm{T}
		\end{split}
	\end{equation*}
\end{proof}

\begin{itembox}[l]{Exercise 2.12.7.}
	Generate a sample and verify
	\begin{math}
		\nabla_{\bm{x}} \langle \bm{a} , \bm{x} \rangle = \bm{a}
	\end{math}
	in Exercise 2.12.6.
\end{itembox}

\begin{proof}
	Let
	\begin{math}
		\bm{a} = \lbrack 1 , 2 \rbrack^{\mathrm{T}}
	\end{math}
	and
	\begin{math}
		\bm{x} \in \mathbb{R}^2 .
	\end{math}
	\begin{equation*}
		\begin{split}
			\nabla_{\bm{x}} \langle \bm{a} , \bm{x} \rangle &=
			\begin{bmatrix}
				\nabla_{x_1} \langle \bm{a} , \bm{x} \rangle \\
				\nabla_{x_2} \langle \bm{a} , \bm{x} \rangle
			\end{bmatrix} \\
			&=
			\begin{bmatrix}
				\nabla_{x_1} (x_1 + 2 x_2) \\
				\nabla_{x_2} (x_1 + 2 x_2)
			\end{bmatrix} \\
			&=
			\begin{bmatrix}
				1 \\
				2
			\end{bmatrix} \\
			&= \lbrack 1 , 2 \rbrack^{\mathrm{T}} \\
			&= \bm{a}
		\end{split}
	\end{equation*}
	We obtain that $\nabla_{\bm{x}} \langle \bm{a} , \bm{x} \rangle = \bm{a}$ for this sample.
\end{proof}


\begin{itembox}[l]{Exercise 2.12.9.}
	Let
	\begin{math}
		\bm{x} , \bm{b} \in \mathbb{R}^n .
	\end{math}
	Show that
	\begin{equation*}
		\nabla_{\bm{x}} \lVert \bm{x} + \bm{b} \rVert^2 = 2 (\bm{x} + \bm{b}) .
	\end{equation*}
\end{itembox}


\begin{proof}
	Let
	\begin{math}
		\bm{x} , \bm{b} \in \mathbb{R}^n .
	\end{math}
	\begin{equation*}
		\begin{split}
			\nabla_{\bm{x}} \lVert \bm{x} + \bm{b} \rVert^2 &=
			\begin{bmatrix}
				\nabla_{x_1} \lVert \bm{x} + \bm{b} \rVert^2 \\
				\vdots \\
				\nabla_{x_n} \lVert \bm{x} + \bm{b} \rVert^2
			\end{bmatrix} \\
			&=
			\begin{bmatrix}
				\nabla_{x_1} \left( \sum_{i=1}^n (x_i + b_i)^2 \right) \\
				\vdots \\
				\nabla_{x_n} \left( \sum_{i=1}^n (x_i + b_i)^2 \right)
			\end{bmatrix} \\
			&=
			\begin{bmatrix}
				\nabla_{x_1} (x_1 + b_1)^2 \\
				\vdots \\
				\nabla_{x_n} (x_n + b_n)^2
			\end{bmatrix} \\
			&=
			\begin{bmatrix}
				\nabla_{x_1} (x_1^2 + b_1^2 + 2 x_1 b_1) \\
				\vdots \\
				\nabla_{x_n} (x_n^2 + b_n^2 + 2 x_n b_n)
			\end{bmatrix} \\
			&=
			\begin{bmatrix}
				2 (x_1 + b_1) \\
				\vdots \\
				2 (x_n + b_n)
			\end{bmatrix} \\
			&= 2 \left(
			\begin{bmatrix}
				x_1 \\
				\vdots \\
				x_n
			\end{bmatrix}
			+
			\begin{bmatrix}
				b_1 \\
				\vdots \\
				b_n
			\end{bmatrix} \right) \\
			&=
			2(\bm{x} + \bm{b})
		\end{split}
	\end{equation*}
\end{proof}

\begin{itembox}[l]{Exercise 2.12.11.}
	Let
	\begin{math}
		\bm{x} , \bm{a} , \bm{b} \in \mathbb{R}^n .
	\end{math}
	Show that
	\begin{equation*}
		\nabla_{\bm{x}} \langle \bm{a} , \bm{x} + \bm{b} \rangle = \bm{a} .
	\end{equation*}
\end{itembox}


\begin{proof}
	Let
	\begin{math}
		\bm{x} , \bm{a} , \bm{b} \in \mathbb{R}^n .
	\end{math}
	\begin{equation*}
		\begin{split}
			\nabla_{\bm{x}} \langle \bm{a} , \bm{x} + \bm{b} \rangle &=
			\begin{bmatrix}
				\nabla_{x_1} \langle \bm{a} , \bm{x} + \bm{b} \rangle \\
				\vdots \\
				\nabla_{x_n} \langle \bm{a} , \bm{x} + \bm{b} \rangle
			\end{bmatrix} \\
			&=
			\begin{bmatrix}
				\nabla_{x_1} \sum_{i=1}^n a_i (x_i + b_i) \\
				\vdots \\
				\nabla_{x_n} \sum_{i=1}^n a_i (x_i + b_i)
			\end{bmatrix} \\
			&=
			\begin{bmatrix}
				\nabla_{x_1} \sum_{i=1}^n a_i (x_i + b_i) \\
				\vdots \\
				\nabla_{x_n} \sum_{i=1}^n a_i (x_i + b_i)
			\end{bmatrix} \\
			&=
			\begin{bmatrix}
				\nabla_{x_1} a_1 (x_1 + b_1) \\
				\vdots \\
				\nabla_{x_n} a_n (x_n + b_n)
			\end{bmatrix} \\
			&=
			\begin{bmatrix}
				a_1 \\
				\vdots \\
				a_n
			\end{bmatrix} \\
			&= \bm{a}
		\end{split}
	\end{equation*}
\end{proof}


\begin{itembox}[l]{Exercise 2.12.13.}
	Let
	\begin{math}
		\bm{x} , \bm{a} \in \mathbb{R}^n .
	\end{math}
	Show that
	\begin{equation*}
		\nabla_{\bm{x}} \langle \bm{a} , \bm{x} \rangle^2 = 2 \langle \bm{a} , \bm{x} \rangle \bm{a} .
	\end{equation*}
\end{itembox}


\begin{proof}
	Let
	\begin{math}
		\bm{x} , \bm{a} \in \mathbb{R}^n .
	\end{math}
	\begin{equation*}
		\begin{split}
			\nabla_{\bm{x}} \langle \bm{a} , \bm{x} \rangle^2 &=
			\begin{bmatrix}
				\nabla_{x_1} \langle \bm{a} , \bm{x} \rangle^2 \\
				\vdots \\
				\nabla_{x_n} \langle \bm{a} , \bm{x} \rangle^2
			\end{bmatrix} \\
			&=
			\begin{bmatrix}
				\nabla_{x_1} (\sum_{i=1}^n a_i x_i)^2 \\
				\vdots \\
				\nabla_{x_n} (\sum_{i=1}^n a_i x_i)^2
			\end{bmatrix} \\
			&=
			\begin{bmatrix}
				2 (\sum_{i=1}^n a_i x_i) a_1 \\
				\vdots \\
				2 (\sum_{i=1}^n a_i x_i) a_n
			\end{bmatrix} \\
			&= 2 \left( \sum_{i=1}^n a_i x_i \right)
			\begin{bmatrix}
				a_1 \\
				\vdots \\
				a_n
			\end{bmatrix} \\
			&= 2 \langle \bm{a} , \bm{x} \rangle
			\begin{bmatrix}
				a_1 \\
				\vdots \\
				a_n
			\end{bmatrix} \\
			&= 2 \langle \bm{a} , \bm{x} \rangle \bm{a}
		\end{split}
	\end{equation*}
\end{proof}


\section{Derivatives of Vector-Valued Functions}

\subsection{Exercise}

\begin{itembox}[l]{Exercise 2.13.2.}
	Let
	\begin{math}
		\bm{a} , \bm{b} \in \mathbb{R}^n .
	\end{math}
	Let a vector-valued function $\bm{f}: \mathbb{R} \rightarrow \mathbb{R}^n$ be defined as
	\begin{equation*}
		\bm{f}(x) = \lbrack \sin(b_1 x) , \sin(b_2 x) , \ldots , \sin(b_n x) \rbrack^\mathrm{T} .
	\end{equation*}
	Show that
	\begin{equation*}
		\nabla_x \langle \bm{a} , \bm{f}(x) \rangle = \sum_{i=1}^n a_i b_i \cos(b_i x) .
	\end{equation*}
\end{itembox}


\begin{proof}
	Let
	\begin{math}
		x \in \mathbb{R}
	\end{math}
	and
	\begin{math}
		\bm{a} , \bm{b} \in \mathbb{R}^n .
	\end{math}
	\begin{equation*}
		\begin{split}
			\nabla_x \langle \bm{a} , \bm{f}(x) \rangle
			&= \nabla_x \left( \sum_{i=1}^n a_i \sin(b_i x) \right) \\
			&= \sum_{i=1}^n a_i \cos(b_i x) b_i \\
			&= \sum_{i=1}^n a_i b_i \cos(b_i x)
		\end{split}
	\end{equation*}
\end{proof}


\begin{itembox}[l]{Exercise 2.13.4.}
	Let
	\begin{math}
		\bm{f}: \mathbb{R} \rightarrow \mathbb{R}^n
	\end{math}
	and
	\begin{math}
		\bm{g}: \mathbb{R} \rightarrow \mathbb{R}^n .
	\end{math}
	Show that
	\begin{equation}
		\label{ex2134}
		\nabla_x \langle \bm{f}(x) , \bm{g}(x) \rangle
		= \langle \nabla_x \bm{f}(x) , \bm{g}(x) \rangle
		+ \langle \bm{f}(x) , \nabla_x \bm{g}(x) \rangle .
	\end{equation}
\end{itembox}


\begin{proof}
	Let
	\begin{math}
		x \in \mathbb{R}.
	\end{math}
	\begin{equation*}
		\begin{split}
			\nabla_x \langle \bm{f}(x) , \bm{g}(x) \rangle &= \nabla_x \left( \sum_{i=1}^n f_i(x) g_i(x) \right) \\
			&= \sum_{i=1}^n (\nabla_x f_i(x) g_i(x) + f_i(x) \nabla_x g_i(x)) \\
			&= \sum_{i=1}^n (\nabla_x f_i(x)) g_i(x) + \sum_{i=1}^n f_i(x) (\nabla_x g_i(x)) \\
			&= \langle \nabla_x \bm{f}(x) , \bm{g}(x) \rangle + \langle \bm{f}(x) , \nabla_x \bm{g}(x) \rangle
		\end{split}
	\end{equation*}
\end{proof}

\end{document}
